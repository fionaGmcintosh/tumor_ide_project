\documentclass[11pt]{article}
\usepackage[utf8]{inputenc}
\usepackage[margin=1in]{geometry}
\usepackage{amsmath}
\usepackage{amsfonts}
\usepackage{amssymb}
\usepackage{booktabs}
\usepackage{graphicx}
\usepackage{hyperref}
\usepackage{listings}
\usepackage{xcolor}

\title{Common Tumor Growth Models Comparison: ODE vs IDE Analysis}
\author{Tumor IDE Project}
\date{\today}

\begin{document}

\maketitle

\section{Abstract}

This project implements a comprehensive comparison between classical Ordinary Differential Equation (ODE) models and Impulsive Differential Equation (IDE) models for tumor growth with radiation therapy, based on the methodology from Laleh et al. (2022). The analysis demonstrates that IDE models, which incorporate discrete radiation therapy sessions as impulses, consistently outperform ODE models, which model treatment as continuous effects on growth parameters, when fitted to real patient data.




\section{Introduction}

Mathematical modeling of tumor growth has become crucial for understanding cancer progression and optimizing treatment strategies. Classical models have been widely used in the field, are easy to implement, and are often used for predicting tumor growth and response to treatments.\n

As mentioned in Laleh et al. (2022), there are two ways in which to implement the effects of therapies when using mathematical models. One is an implicit way, where the effects are solely seen through their alterations of the parameters in the equations. This was the method chosen by Laleh et al. The other method, however, explicitly takes into account the actions of the therapy. In this case, at the moment that treatment is given, the acute change in volume (i.e. the acute death of certain parts of the tumor when treatment is administered) is made as an impulse, then the model returns to its continuous state until the next impulse of treatment.\n

This project addresses this gap by implementing and comparing six classical tumor growth models in both ODE and IDE formulations.

\subsection{Research Question}

Does modeling radiation therapy as discrete impulses (IDE) provide better predictive accuracy than modeling it as a continuous effect (ODE) when fitted to real patient data?




\section{Brief Literature Review}
This is where the literature review will go. For now, just leaving all the stuff in the "sources to include" thing below to collect.

\section{Sources to includes}
\begin{enumerate}
	\item https://journals.plos.org/ploscompbiol/article?id=10.1371/journal.pcbi.1009822
\end{enumerate}




\section{Methodology}

The methods required data collection, model implementation, and computational experimentation.

\subsection{Data}
Data was gathere from ***put wherever we got the data from [NYU etc etc from year - year*** RECIST data. There were ## studies from which the data was gathered. These included trials of ***drugs/types*** for ***cancer types***.\n

This data did not include the exact dates/timeline for the treatment. This is a limitation that is discussed later. However, RECIST measurement data is most often collected either the same day or the day before treatment is given for a cycle. It can sometimes be as far as 7 days after the measurement is taken. Thus, to standardize, we chose to make the assumption that treatment was given one day after the measurement was taken. So if the time point is day $X$, the treatment is assumed to be day $X+1$.

The data was fully deidentified with an encryption key to reidentify details as needed. 

\begin{table}[h] % 'h' means place table approximately here
\centering
\begin{tabular}{|c|c|c|c|} % 3 columns, all centered, with vertical lines
\hline
\textbf{ID} & \textbf{Treatment Group} & \textbf{Days Since Baseline} & \textbf{Longest Diameter} \\ \hline
1 & Alice & 85 & etc\\ \hline
2 & Bob & 90 & etc\\ \hline
3 & Charlie & 78 & etc\\ \hline
\end{tabular}
\caption{Example of the deidentified data}
\label{tab:simple}
\end{table}

Standard way to approximate volume is to say $V=\frac{LD^3}{2}$, where $LD$ is the longest diameter measured.

\subsection{Models Implemented}

Six classical tumor growth models were implemented:

\begin{enumerate}
    \item \textbf{Exponential}: $\frac{dV}{dt} = rV$
    \item \textbf{Logistic}: $\frac{dV}{dt} = rV(1 - \frac{y}{K})$
    \item \textbf{Classic Bertalanffy}: $\frac{dV}{dt} = aV^{2/3} - bV$
    \item \textbf{General Bertalanffy}: $\frac{dV}{dt} = aV^m - bV^n$
    \item \textbf{Classic Gompertz}: $\frac{dV}{dt} = ry\ln(\frac{K}{V})$
    \item \textbf{General Gompertz}: $\frac{dV}{dt} = ry\ln(\frac{K}{V})^{1/m}$
\end{enumerate}

\subsection{ODE vs IDE Formulations}

The same six models will be used for both the ODE and IDE forms. The specifics of how this will look are seen below.

\subsubsection{ODE Models}
ODE models incorporate treatment as a continuous effect on growth parameters. Thus, the six methods seen above are incorporated precisely as seen, with the differences in the estimation of parameters such as $r$, $a$, $b$, etc. being the exclusive way that the treatment affects the growth prediction.

\subsubsection{IDE Models}
IDE models incorporate treatment as discrete impulses at treatment times:
\begin{align}
\frac{dV}{dt} &= f(t, V, \text{params}) \quad \text{[between treatments]} \\
V(t^+) &= V(t^-) \cdot (1 - \delta) \quad \text{[at treatment times]}
\end{align}
Where $\delta$ is a parameter correlating to the death of the tumor cells caused acutely by the treatment at the time of treatment.

\subsection{Two Experiments}

Following Laleh et al. (2022), two experiments were conducted:

\subsubsection{Experiment 1: Goodness of Fit}
\begin{itemize}
    \item Use ALL available data points from each patient
    \item Fit both ODE and IDE models to complete dataset
    \item Measure Root Mean Square Error (RMSE) against observed data
    \item Question: "Which model best describes the data we have?"
\end{itemize}

\subsubsection{Experiment 2: Early Prediction}
\begin{itemize}
    \item Use only first 50\% of data points for fitting
    \item Predict remaining 50\% of data points
    \item Measure Mean Absolute Error (MAE) for predictions
    \item Question: "Which model best predicts future outcomes from early data?"
\end{itemize}




\section{***Results***}

***ALL OF THIS IS IS CURRENTLY PHONY DATA! Just using it as a template sort of idea for right now.

\subsection{Visualization of Model Predictions}

Figure \ref{fig:model_comparison} shows the experimental data overlaid with both ODE and IDE model predictions for all six classical tumor growth models. The visualization clearly demonstrates:

\begin{itemize}
    \item \textbf{Experimental Data} (black dots): Synthetic patient measurements with realistic noise
    \item \textbf{ODE Predictions} (red dashed lines): Continuous treatment modeling
    \item \textbf{IDE Predictions} (blue solid lines): Discrete impulse modeling
    \item \textbf{Treatment Times} (gray vertical lines): Radiation therapy sessions
\end{itemize}

The plots show that IDE models (blue lines) consistently follow the experimental data more closely than ODE models (red dashed lines), particularly during the treatment period (days 20-76).

\begin{figure}[h]
\centering
\includegraphics[width=0.9\textwidth]{graphs/tumor_model_comparison}
\caption{Model Comparison: Experimental data vs ODE vs IDE predictions for all six classical tumor growth models. IDE models (blue solid lines) show better fit to experimental data (black dots) compared to ODE models (red dashed lines). Gray vertical lines indicate treatment times.}
\label{fig:model_comparison}
\end{figure}

\subsection{Experiment 1: Goodness of Fit}

Table \ref{tab:exp1} shows the RMSE results for all models when fitted to complete patient data.

\begin{table}[h]
\centering
\caption{Experiment 1 - Goodness of Fit Results (RMSE)}
\label{tab:exp1}
\begin{tabular}{@{}lcc@{}}
\toprule
Model & ODE RMSE & IDE RMSE \\
\midrule
Exponential & 807.01 & 103.61 \\
Logistic & 646.27 & 94.08 \\
Classic Bertalanffy & 35.19 & 10.49 \\
General Bertalanffy & 45.79 & 11.75 \\
Classic Gompertz & 2432.01 & 1448.98 \\
General Gompertz & 1872.24 & 796.27 \\
\bottomrule
\end{tabular}
\end{table}

\textbf{Result:} IDE models win 6/6 models (100\% better fit)

\subsection{Experiment 2: Early Prediction}

Table \ref{tab:exp2} shows the MAE results for early prediction using only first half of data.

\begin{table}[h]
\centering
\caption{Experiment 2 - Early Prediction Results (MAE)}
\label{tab:exp2}
\begin{tabular}{@{}lcc@{}}
\toprule
Model & ODE MAE & IDE MAE \\
\midrule
Exponential & 1022.70 & 88.94 \\
Logistic & 835.43 & 78.38 \\
Classic Bertalanffy & 44.57 & 5.66 \\
General Bertalanffy & 58.49 & 4.61 \\
Classic Gompertz & 3215.91 & 1689.20 \\
General Gompertz & 2439.36 & 898.66 \\
\bottomrule
\end{tabular}
\end{table}

\textbf{Result:} IDE models win 6/6 models (100\% better prediction)




\section{***Discussion***}

***ALL OF THIS IS IS CURRENTLY PHONY DATA! Just using it as a template sort of idea for right now.

\subsection{Key Findings}

\begin{enumerate}
    \item \textbf{IDE models consistently outperform ODE models} across all six classical tumor growth models
    \item \textbf{Discrete treatment modeling} (IDE) is superior to continuous treatment modeling (ODE)
    \item \textbf{Early prediction is challenging} - early treatment response shows only moderate correlation with final response
    \item \textbf{Clinical relevance} - IDE models better represent real radiation therapy practice
\end{enumerate}

\subsection{Clinical Implications}

The results support the use of IDE models for:
\begin{itemize}
    \item Treatment planning and optimization
    \item Outcome prediction from early data
    \item Clinical decision-making
    \item Personalized medicine approaches
\end{itemize}

\subsection{Model Performance}

The best performing models were:
\begin{enumerate}
    \item Classic Bertalanffy (IDE): RMSE = 10.49, MAE = 5.66
    \item General Bertalanffy (IDE): RMSE = 11.75, MAE = 4.61
    \item Logistic (IDE): RMSE = 94.08, MAE = 78.38
\end{enumerate}




\section{***Conclusion***}

***ALL OF THIS IS IS CURRENTLY PHONY DATA! Just using it as a template sort of idea for right now.

This analysis validates the methodology from Laleh et al. (2022) and demonstrates that:

\begin{enumerate}
    \item IDE models provide superior predictive accuracy compared to ODE models
    \item Discrete treatment modeling better represents clinical radiation therapy
    \item Classical tumor growth models can be effectively validated against patient data
    \item Early prediction capabilities vary significantly between models
    \item Amongst the ODEs, ### model was the best at prediction and ### was the best fittable model
    \item Amongst the IDEs, ### model was the best at prediction and ### was the best fittable model
    \item Overall ### was best at prediction and ### was best able to be fit
\end{enumerate}

The results support the continued development and clinical application of IDE-based tumor growth models for radiation therapy planning and outcome prediction.




\section{References}

Laleh, N. G., Loeffler, C. M. L., Grajek, J., Staňková, K., Pearson, A. T., Muti, H. S., ... \& Kather, J. N. (2022). Classical mathematical models for prediction of response to chemotherapy and immunotherapy. \textit{PLOS Computational Biology}, 18(2), e1009822.




\section{Code Availability}

All code and analysis scripts are available in the GitHub repository (which we'll need to link to this).
\end{document}
